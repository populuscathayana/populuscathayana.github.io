% 文档类声明,指定了使用 article 类型和 A4 纸张大小。
\documentclass[a4paper]{article}
\usepackage{expl3}

% 引入各种数学符号和数学排版工具。
\usepackage{amsmath}
% 用于更复杂的表格排版。
\usepackage{array}
% 提供更多的数学符号。
\usepackage{amssymb}
% 用于标题目录的自定义样式。
\usepackage{titletoc}
% 用于标题样式的自定义。
\usepackage{titlesec}
% 计算长度,通常用于布局设计。
\usepackage{calc}
% 提供更多关于浮动体的控制选项。
\usepackage{float}
% 自定义图表和表格标题的样式。
\usepackage{caption}
% 提供子图表的支持。
\usepackage{subcaption}
% 提供书签功能,与 hyperref 宏包协同工作。
\usepackage{bookmark}
% 设置页面大小和边距。
\usepackage{geometry}
% 提供超链接功能。
\usepackage{hyperref}
% XeLaTeX专用宏包,用于字体选择等。
\usepackage{xltxtra,fontspec,xunicode}
% 使用 Type1 字体编码。
\usepackage{type1cm}
% 提供用于花体手写字体的支持。
\usepackage{mathrsfs}
% 另一种手写体风格的宏包。
\usepackage{calligra}
% 用于设置背景的宏包。
\usepackage{background}
% 引入图像和处理图片的宏包。
\usepackage{graphicx}
% 使用现代字体。
\usepackage{lmodern}
% 设置字体编码为 T1。
\usepackage[T1]{fontenc}
% 引入哥特花体宏包。
\usepackage{yfonts}
% 引入另一个花体字体宏包。
\usepackage{aurical}
% 用于生成填充文本。
\usepackage{lipsum}
% 使用 Times 字体的数学字体版本。
\usepackage{mathptmx}
%为字体添加阴影
\usepackage{shadowtext}
% 用于旋转对象
\usepackage{rotating} 
% 中文断行设置。
\XeTeXlinebreaklocale "zh"
\XeTeXlinebreakskip = 0pt plus 1pt minus 0.1pt
% xeCJK 宏包设置,允许斜体和粗体。
\usepackage[slantfont,boldfont]{xeCJK}
% 设置默认的中文字体为黑体。
\setCJKmainfont{Hei}
% 设置等宽字体为黑体。
\setCJKmonofont{Hei}
% 设置默认的英文衬线字体为 Arial。
\setmainfont{Arial}
% 设置默认的英文等宽字体为 Monaco。
\setmonofont{Monaco}
% 设置默认的英文无衬线字体为 Trebuchet MS。
\setsansfont{Trebuchet MS}
% 自定义字号命令。
\newcommand{\yihao}{\fontsize{26pt}{36pt}\selectfont}           % 一号, 1.4 倍行距
\newcommand{\erhao}{\fontsize{22pt}{28pt}\selectfont}          % 二号, 1.25倍行距
\newcommand{\xiaoer}{\fontsize{18pt}{18pt}\selectfont}          % 小二, 单倍行距
\newcommand{\sanhao}{\fontsize{16pt}{24pt}\selectfont}        % 三号, 1.5倍行距
\newcommand{\xiaosan}{\fontsize{15pt}{22pt}\selectfont}        % 小三, 1.5倍行距
\newcommand{\sihao}{\fontsize{14pt}{21pt}\selectfont}            % 四号, 1.5 倍行距
\newcommand{\banxiaosi}{\fontsize{13pt}{19.5pt}\selectfont}    % 半小四, 1.5倍行距
\newcommand{\xiaosi}{\fontsize{12pt}{18pt}\selectfont}            % 小四, 1.5倍行距
\newcommand{\dawuhao}{\fontsize{11pt}{11pt}\selectfont}       % 大五号, 单倍行距
\newcommand{\wuhao}{\fontsize{10.5pt}{15.75pt}\selectfont}    % 五号, 单倍行距


% 设置文档的边距。
\geometry{left=3.18cm,right=3.18cm,top=2.54cm,bottom=2.54cm}
% 设置节标题的字体大小。
\titleformat*{\section}{\fontsize{14pt}{21pt}}
\titleformat*{\subsection}{\fontsize{12pt}{18pt}}
\titleformat*{\subsubsection}{\fontsize{12pt}{18pt}}
\title{\xiaosan 团簇}  
% 设置文章标题、作者、日期。
\author{cathayana(3210100360),指导老师:xxx}
\date{}

% 重新定义摘要名称和目录名称。
\renewcommand{\abstractname}{}
% 自定义目录项的样式。
\renewcommand{\contentsname}{\centerline{目\quad 录}}
\titlecontents{section}[2em]{\bfseries \wuhao \vspace{10pt}}{\contentslabel{2em}}{\hspace*{-2em}}{~\titlerule*[0.6pc]{$.$}~\contentspage}
\titlecontents{subsection}[4em]{\wuhao}{\contentslabel{2em}}{\hspace*{-2em}}{~\titlerule*[0.6pc]{$.$}~\contentspage}
\titlecontents{subsubsection}[7em]{\wuhao}{\contentslabel{3em}}{\hspace*{-2em}}{~\titlerule*[0.6pc]{$.$}~\contentspage}
\titlecontents{paragraph}[11em]{\wuhao}{\contentslabel{4em}}{\hspace*{-2em}}{~\titlerule*[0.6pc]{$.$}~\contentspage}

% 超链接设置,隐藏链接颜色和边框。
\hypersetup{hidelinks}

% 设置背景图像
\backgroundsetup{
scale=1,
color=white,
opacity=1,
angle=0,
contents={%
 \includegraphics[width=\paperwidth,height=\paperheight]{./image/yang.png}
  }
}


% 定义一个新命令用于竖排文本
\newcommand{\verticaltext}[1]{%
  \begingroup % 开始局部分组
  \def\arraystretch{0.05} % 设置行间距为0
  \setlength{\tabcolsep}{0pt} % 设置列间距为0
  \begin{tabular}[t]{@{}c@{}} % 开始无间距的表格
  \verticalletters#1\end%
  \end{tabular}%
  \endgroup % 结束局部分组
}

% 定义一个新命令用于逐字母输出
\def\verticalletters#1{%
  \ifx#1\end% 检查是否结束
    \let\next\relax% 结束递归
  \else%
    \scalebox{1}[-1]{#1}\\% 输出当前字母并垂直翻转
    \let\next\verticalletters% 继续处理下一个字母
  \fi%
  \next% 继续递归处理或结束
}


% 定义一个新命令用于拆分并竖排文本
\newcommand{\verticalsentence}[1]{%
  \verticalwords#1 \end
}

% 定义一个新命令用于递归处理每个单词
\def\verticalwords#1 {%
  \ifx\end#1% 检查是否结束
    \let\next\relax% 结束递归
  \else%
    \verticaltext{#1}\hspace{-2em}\quad% 输出当前单词的竖排文本并添加间隔
    \let\next\verticalwords% 继续处理下一个单词
  \fi%
  \next% 继续递归处理或结束
}
