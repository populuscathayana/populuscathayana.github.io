% 文档内容开始
\begin{document}
\Huge
% 使用mathrsfs宏包的花体字母
 $\mathscr{ABCDEF}$.

% 使用calligra宏包的花体字母
{\calligra ABCDEF}.

{\gothfamily
Text in Fraktur Blackletter for the entire document.
}

% 切换到哥特字体
\textgoth{Text in Gothic Blackletter.}

% 切换到斯瓦比亚哥特字体
\textswab{Text in Swabian Blackletter.}

% 切换到弗拉克图哥特字体
\textfrak{Text in Fraktur Blackletter.}

\shadowtext{\textfrak{Text in Fraktur Blackletter.}}

{\Fontauri This is an example of Auriocus Kalligraphicus.}
\shadowtext{
\fontspec{Hei}{好wed}
\fontspec{黑体}{好weq}
\fontspec{幼圆}{好wead}
}

\fontspec{longmai-Regular}{好wead}

%使用分单词输出需要给出参数\hspace{-1.5em}
\fontspec{longmai}{
\verticaltext{Hello}\hspace{-1.5em}
\verticaltext{World.}\hspace{-1.5em}
\verticaltext{This}\hspace{-1.5em}
\verticaltext{is}\hspace{-1.5em}
\verticaltext{vertical}\hspace{-1.5em}
\verticaltext{text!}
}


%使用句子自动分割输出的已经自动给出参数,注意参数仅适用于当前字体longmai-Regular或者longmai
\fontspec{longmai}{
\verticalsentence{No matter how many times I attempt to reach it, the outcome will never change. But still I continue to fight.}
}


\shadowtext{
{\Fontauri This is an example of Auriocus Kalligraphicus.}
}
\normalsize
\newpage
\begin{figure}[ht]
  \centering
  % 第一张子图
  \begin{subfigure}[b]{0.4\textwidth}
    \includegraphics[width=\textwidth]{./image/eda/eda1.jpg}
    \caption{Eda1}
    \label{fig:sub1}
  \end{subfigure}
  \hfill % 在子图之间添加一些水平空间
  % 第二张子图
  \begin{subfigure}[b]{0.4\textwidth}
    \includegraphics[width=\textwidth]{./image/eda/eda2.jpg} 
    \caption{Eda2}
    \label{fig:sub2}
  \end{subfigure}
  % 如果你有更多的子图,可以继续添加更多的subfigure环境
  \caption{两张图片的示例}
  \label{fig:test}
\end{figure}


\end{document}